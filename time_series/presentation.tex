\documentclass[10pt]{beamer}

\usetheme[progressbar=frametitle]{metropolis}
\usepackage{appendixnumberbeamer}

\usepackage{booktabs}
\usepackage[scale=2]{ccicons}

\usepackage{pgfplots}
\usepgfplotslibrary{dateplot}

\usepackage{xspace}
\newcommand{\themename}{\textbf{\textsc{metropolis}}\xspace}

\title{Solar Power Savings}
\subtitle{A Time Series Analysis}
% \date{\today}
\date{}
\author{Kaylee Hodgson and Michael Christiansen}
\institute{Brigham Young University}
% \titlegraphic{\hfill\includegraphics[height=1.5cm]{logo.pdf}}

\begin{document}

\maketitle

\begin{frame}{Table of contents}
  \setbeamertemplate{section in toc}[sections numbered]
  \tableofcontents[hideallsubsections]
\end{frame}

\section{Introduction}

\begin{frame}[fragile]{Background}
\begin{itemize}
\item Solar energy sources have become an excellent, environmentally friendly alternative to fossil fuels, especially with the increasing demand for energy in a modern and highly industrialized society.
\item Solar energy is particularly appealing for homeowners who can, in the long run, save money by using solar energy instead of making monthly payments to a power company that uses fossil fuels.
\end{itemize}
\end{frame}

\begin{frame}{Motivation}
\begin{enumerate}
\item Sales Tactics
\begin{itemize}
\item Solar companies look for sales tactics to convince customers to buy solar panels for their homes.
\item The upfront cost for solar panels is large, which makes it difficult to convince customers that solar panels could end up saving them money. 
\end{itemize}
\item Customer Savings
\begin{itemize}
\item Customers would benefit from being able to more accurately budget for the cost of power
\end{itemize}
\end{enumerate}
\end{frame}

\begin{frame}[fragile]{Questions of Interest}
\begin{enumerate}
\item How much per month do customers with solar panels save on average compared to those without?
\item How much time on average will it take for a customer who has bought solar panels to recoup the initial cost (in this case \$8,000)?
\end{enumerate}
\end{frame}

\begin{frame}{Goal of Analysis}
\begin{enumerate}
\item Inference: Determine how much the customer will save on average by switching to solar
\item Prediction: Predict what bills are going to be in the future to determine how long it will take to recoup the \$8,000
\end{enumerate}
\end{frame}

\section{Exploring the Data}

\begin{frame}{The Data}
\itemize
\item This dataset has 51 sequential observations of the monthly power bills for a single customer, along with an indicator for whether the customer had solar panels that month or not. 
\item The first 29 months the customer did not have solar panels. The last 22 months, they did.
\end{frame}

\begin{frame}{The Data}
\begin{figure}
\begin{center}
\caption{Customer's Monthly Power Bill}
\includegraphics[width=7cm,height=5cm]{eda1.png}
\label{splot}
\smallskip
\end{center}
\end{figure} 
The data is correlated in time, so a simple linear regression model, which assumes independence of the observations, will give inaccurate uncertainty measurements.
\end{frame}

\section{Methods}

\begin{frame}{Correlated Generalized Least Squares Model}
Generalized Least Squares does not assume a distribution
\begin{equation}
Y \sim N(X\beta,\sigma^2R) \Rightarrow Y \sim N(X\beta,\Sigma) ,
\end{equation}
where $R$ depends on the covariance structure
\smallskip
We estimate the MLE's of $\hat{\beta}$ and the $\hat{\sigma}^2$ as follows:
\begin{equation}
\hat{\beta} = (X'R^{-1}X)^{-1}X'R^{-1}Y
\end{equation}
\begin{equation}
\hat{\sigma}^2=\frac{1}{N}(Y-X\beta)'R^{-1}(Y-X\beta)
\end{equation}
\end{frame}

\begin{frame}{Covariance Structures}
\begin{itemize}
\item Lag-1 Autoregressive Process: adjusts the correlation value according to the distance of the observations, decreasing the correlation as the time between two observations increases
\item Moving Average Process: only accounts for correlation between consecutive time points
\item Exponential Correlation: can be used for unequally spaced time periods
\end{itemize}
Because our data is equally spaced and we suspect that correlation decreases as time between observations increases, we choose the Lag-1 Autoregressive Process.
\end{frame}

\begin{frame}{Lag 1 Autoregressive}
Covariance Structure
\begin{equation}
R = \sigma^2
\begin{bmatrix} 
1 \quad \phi \quad \phi^2 \quad \hdots \quad \phi^{T-1} \\ 
\phi \quad 1 \quad \phi \quad \hdots \quad \phi^{T-2} \\ 
\vdots \quad \quad \vdots \quad \quad \ddots \quad \quad \vdots \quad \quad \vdots \\
\phi^{T-1} \quad \hdots \quad \hdots \quad \hdots \quad 1 \\
\end{bmatrix}
\end{equation}
\end{frame}

\begin{frame}{Model Variables}
\itemize
\item In our model, we decided to include fixed effects for whether a customer is using solar power, and interaction terms between power type and the seasons of Summer and Winter
\item These interactions were included due to seasonal variability in the efficiency of different power types.
\item Our covariance structure is based on time between observations
\end{frame}

\begin{frame}{Gaussian Process Regression Assumptions}
\begin{itemize}
\item Data is multivariate normal - residuals should be normally distributed
\item Constant variance
\end{itemize}
To check the assumptions of the model, we used Cholesky Decomposition to decorrelate the residuals
\begin{figure}
\begin{center}
\includegraphics[width=8cm,height=4cm]{modas.pdf}
\smallskip
\caption{Model Assumptions Check}
\end{center}
\end{figure} 
\end{frame}

\begin{frame}{Model Fit}
Results from Cross-Validation Study
\begin{itemize}
\item Mean Bias = -0.557
\item RPMSE = 34.38
\item Mean Prediction Interval Coverage = 0.842
\item Mean Prediction Interval Width = 104.098
\end{itemize}
Percentage of the Variance Explained Using the Decorrelated Residuals
\itemize
\item R-squared = 0.77
\end{frame}

\section{Results}

\begin{frame}{Data with Model Fitted}
\begin{figure}
\begin{center}
\includegraphics[width=8cm,height=6cm]{fitted.png}
\smallskip
\caption{Model Fitted to Power Bill Data}
\end{center}
\end{figure} 
\end{frame}

\begin{frame}{Model Coefficients}
\begin{table}[H]
\begin{center}
\caption{AR(1) Model Coefficient Estimates}
\begin{tabular}{l | l c c }
\hline
Coefficient &  Estimate  &   95\% CI & p-value \\
\hline
(Intercept)  &    110.241309 & (93.25474,127.22788) & 0.0000 \\
SolarY       & -87.910388 & (-113.49516,-62.32561) & 0.0000 \\
SolarN:Winter &  2.642468 & (-23.84270,29.12763) & 0.8416 \\
SolarY:Winter &  74.717153 & (41.06610,108.36821) & 0.0001 \\
SolarN:Summer &  70.056139 & (39.57363, 100.53865) & 0.0000 \\
SolarY:Summer &  8.863714 & (-22.76523,40.49266) & 0.5753 \\
\hline
\end{tabular}
\end{center}
\end{table}
\begin{table}[H]
\begin{center}
\caption{AR(1) Model Variance Component Estimates}
\begin{tabular}{l | l c }
\hline
 &  Estimate  &   95\% CI \\
\hline
$\sigma$ & 28.08791 & (23.07942,34.18329) \\
$\phi$ & 0.1006577 & (-0.2019074,0.3856839) \\
\hline
\end{tabular}
\end{center}
\end{table}
\end{frame}

\begin{frame}{Inference}
A Customer saves \$1022.28 on average per year from switching to Solar, approximately \$85.19 per month
\begin{figure}
\begin{center}
\includegraphics[width=7cm,height=5cm]{averagesavings.pdf}
\smallskip
\caption{Average Savings Each Month of the Year from Switching to Solar}
\end{center}
\end{figure} 
\end{frame}

\begin{frame}{Prediction}
It will take a customer on average just over 8 years (96.4 months) to recoup the \$8,000, which a 95\% prediction interval of 91 months to 104 months.
\begin{figure}
\begin{center}
\includegraphics[width=7cm,height=5cm]{money_saved.png}
\smallskip
\caption{100 Samples from Simulation Calculating amount saved with Time. Red Line at \$8000 corresponds to installation costs of Solar Panels.}
\end{center}
\end{figure} 
\end{frame}

\begin{frame}{Conclusions}
\itemize
\item We find how much on average a customer will save by switching to solar, and predict how long it will take to recoup the initial \$8,000 cost of the solar panels
\item These findings inform both the sales tactics of solar companies and the customer's budgeting for the initial cost of the panels and the subsequent monthly payments.
\end{frame}

\begin{frame}{Shortcomings and Future Research}
Shortcomings of the Model:
\begin{itemize}
\item Doesn't account for unequal time periods
\item Model doesn't fully capture extreme data values
\end{itemize}
Future Research:
\begin{itemize}
\item This dataset had only the power bills for one customer living in Provo, UT. The model should be fit using additional customers to determine if these findings are generalizable
\end{itemize}
\end{frame}

\end{document}



